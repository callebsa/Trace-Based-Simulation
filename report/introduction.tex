\chapter{Introduction}
\label{chap:In}
 \section{Motivation}
 \label{sec:motivation}
We are always in the need of faster and smaller computers that burns less energy. In
the second part of the twentieth century the speed of our CPUs (center process unit)
were increased by letting them run faster and faster. The problem is that a linear
frequency speed up causes an exponential growth in energy and heat. A rule of thumb
is that you will be spending as much energy as needed for the computation to cool the
circuit. Since we want to make smaller and smaller units, its gets harder and harder
to cool down. Less materials that are getting warmer will be almost impossible to
cool down after some limit. But we are on the other hand still developing software
that requires more and more of the hardware. When we cannot speed up our CPUs
in the same way as before, we need to see if we can be more effective. Thats when
CPUs introduced pipelining. Lets say that an instruction take six seconds to compute
the we and up with two finished instructions after twelve seconds. If we then divided
the work into three stages where one instruction spends two second each, meaning
that we can begin working on a new instruction every two seconds. Given this, four
instruction (twice the amount) will be done in twelve seconds, this results in increasing
the work that can be done during a period of time without increasing the speed
of the CPU and therefore the energy consumption roughly stays the same.
\\ \\
Another thing to take advantage of is that computing some instructions especially
memory instructions (loads and stores) are involving long waiting times. If we then
instead of just waiting, goes ahead and computes other work, with the next instructions.
We are not effected by the waiting time in the long run because we are starting
the computation of the instruction next in line while waiting. This is called out of
order execution and it is up to the CPU to decide if the work can be done in another
order and still produce the same outcome.
\\ \\
We have sad that the computation of memory instructions include waiting on data
and that we can do useful tasks while waiting. Still it is good if we can decrease the
waiting time, since we might not always have useful tasks to work on that cover the
entire waiting time. In this thesis will look if we can shorter the waiting time by
prefetching the data needed for store instructions in advance. The earlier data is
prefetch the earlier it will be ready to use. This little pieces in the question of making
a CPU do more useful work per unit of energy is what this thesis will be about.

\section{Scope}
\label{sec:scope}
This master thesis will introduce and evaluate three given policies for prefetching data
for load instruction. There are not much research to be found that are focusing on
the acceleration of load instructions in particular. The three basic policies will be
combined in different ways to try to come out with a more optimal policy in terms of
speed up (number of cycles) and power consumption. A bit of the preparation work
in terms of manipulating and understanding traces from benchmarks programs will
be covered as well.
 \section{Related work}
Below you can read about four related works that have in common that they have all
used gems 2.2.3. \fixme

\paragraph{No need to squash and re-execute for reordering} Alberto Ros, Trevor E. Carlson,
Mehdi Alipour and Stefanos Kaxiras, have written the report ”Non-Speculative
Load-Load Reordering in TSO” [12]\fixme. For decades it has been a fact that if a CPU
(in a multi core architecture) reorders a load to gain performance then the other
core must upon observation squashed and re-execute speculative instructions. This
paper can, for the first time, show that the action is not necessary for total store
ordering (TSO). The reordering can instead be hidden for the other cores. This allow
us to commit reordered loads out-of-order without having to wait for the loads to
become non-speculative or without having to checkpoint committed sate and rollback
if needed.There solution is cost-effective and increase the performance of out-of-order
commit by a sizable margin in comparison with the base case where memory operations
where not allowed to commit if the consistency model cold be violated.


\section {Structure of the report}
\label{Sec:struct}

This report consist of seven chapter and one appendix chapter. \fixme
\paragraph{Chapter 1 - Introduction} It gives a motivation of the work along with its scope
and related work.
\paragraph{Chapter 2 - Background} This chapter is divided into two parts. The first covers
the architecture and related terms that will be used throughout the report. The first
part will also cover the three given polices. The second part covers the given softwares
used and benchmarks.
\paragraph{Chapter 3 - Setting Up The Testbed} This can be seen as a continuing of the
second part in chapter 2, here we go through the manipulation of existing tools and
the creating of new ones. The trace which has manipulated to support the simulation
is covered in this chapter.
\paragraph{Chapter 4 - Proposed Prefetch Policies} This will introduce all prefetch polices
that is proposed within the work of this thesis. They are all combinations of the basic
three introduced in the background.
\paragraph{Chapter 5 - Results} Here you will find graphs comparing the different polices
with different settings in terms of power, L0 accesses, useful prefetch and energy.
\paragraph{Chapter 6 - Conclusions} This chapter will try to explain the results and draw
some conclusions of it.
\paragraph{Chapter 7 - Discussion} mandatory??
\paragraph{Chapter A - Implementation details} In this appendix we will cover the key
parts of the implementation of the new policies in some pseudo code examples.