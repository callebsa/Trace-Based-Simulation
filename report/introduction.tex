\chapter{Introduction}
\label{chap:In}
 \THsec{Motivation}{motivation}
We are always in need of faster and smaller computers that burn less energy. In
the second part of the twentieth century the speed of our CPUs (central processing unit)
Was increased by letting them run at higher and higher frequencies. The problem is that a linear
frequency speed-up causes an exponential growth in energy and heat. A rule of thumb
is that you are spending as much energy as needed for the computation to cool the
circuit. Since we want to make smaller and smaller units, it gets harder and harder
to cool the circuits down. However, the software developers still
require more and more of the hardware. When we cannot speed-up our CPUs
in the same way as before, we need to see if we can make the CPU more effective. That is why
CPUs introduced pipelining. Let's say that one instruction takes six seconds to compute
then we end up with two completed instructions after twelve seconds. If we then divide
the work into three stages where one instruction spends two seconds in each stage, this means
that we can begin working on a new instruction every two seconds. Given this improvement, four
instruction (twice the amount) are complete in twelve seconds. The work that can be done during a period without increasing the speed
of the CPU and therefore the energy consumption roughly stays the same.
\\ \\
Another thing to take advantage of is that computing some instructions, especially
memory instructions (loads and stores) involves long waiting times. If we instead of waiting, go ahead and compute other work, with the next instructions,
we would not be affected by the waiting time in the long run. Cause we are starting
the computation of the instruction next in line while waiting. Out of
order execution is the name of it, and it is up to the CPU to decide if the work can be done in another
order and still produce the same outcome.
\\ \\
We have said that the computation of memory instructions include waiting for store permission
and that we can do useful tasks while waiting. Still, it is advantageous if we can decrease the
waiting time since we might not always have useful tasks to work on that cover the
entire waiting time. This thesis aims to investigate if we can shorten the waiting time by
prefetching the data needed for store instructions in advance. The earlier data blocks, the earlier it is ready to use. This little pieces in the question of making
a CPU do more useful work per unit of energy is what this is the aim of this thesis.
 \THsec{Scope}{scope}
This master thesis introduces and evaluates the three state-of-the-art policies \ref{subsec:GPP} for prefetching permission
for the store instruction. There is not much research to be found focusing on
the acceleration of store instructions in particular. The three state-of-the-art policies (OnExecure, OnCommit, and NoPrefetch) are going to be combined in different ways to try to come out with a more optimal policy concerning
speed up (number of cycles), number of prefetches, L1 accesses, and power consumption. Furthermore, some of the preparation work
concerning manipulating and understanding traces from benchmarks programs are covered as well.
 \THsec{Related work}{relWork}
The related work for this master thesis is covered in subsection~\ref{subsec:GPP} were the state-of-the-art store prefetch policies are described.


\THsec{Structure of the report}{structure}

This report consists of eight chapters. 
\paragraph{Chapter \ref{chap:In} - Introduction} This gives a motivation of the work along with its scope. 
\paragraph{Chapter \ref{chap:bg} - Background} This chapter is divided into two parts. The first part covers
the used CPU architecture and related terms to be used throughout the report along with introducing the state-of-the-art policies. The second part covers the methodology employed in this thesis.
\paragraph{Chapter \ref{chap:SettingUpTheTestbed} - Setting Up The Testbed} This chapter can be seen as a continuation of the
second part in chapter 2. Here we go through the modification of existing simulation tools. The trace employed to connect the simulation infrastructure
is covered in this chapter.
\paragraph{Chapter \ref{chap:ProposedPrefetchPolicies} - Proposed Prefetch Policies} This chapter introduces all store prefetch policies
that are proposed within the work of this thesis.
\paragraph{Chapter \ref{chap:results} - Results} In this chapter you find graphs comparing the different policies
with different settings concerning execution times, L1 accesses, useful prefetch, and energy.
\paragraph{Chapter \ref{chap:discussion} - Discussion} Issues with the set up that can have an impact on the results are represented in this chapter.
\paragraph{Chapter \ref{chap:conclusion} - Conclusions} This chapter will offer conclusion about the thesis.
\paragraph{Chapter \ref{chap:fetueredWork} - Future work} This chapter presents ideas on future work that can be built upon the work of this thesis.
