\chapter{Setting Up The Testbed}
\label{chap:SettingUpTheTestbed} 
\THsec{Changes to Sniper}{chSniper}
 In this chapter, there will more focus on the traces, the interface between Sniper
and GEMS simulators. Traces are simply a list of the CPU instructions that will be
simulated, a subject that will be covered more in section \ref{sec:trace}. Sniper produce the
trace which in turn will be used by GEMS. Some minor updates have been done to
the parts of Sniper that print the trace:

\begin{itemize}
\item Writing the start PC (program counter) at first line in the trace.
\item Writing the length of the previous instruction to gather with every instruction.
\item Prints the target PC of a branch together with a ”*” if the branch is taken.
\item Implementing a new flag,\emph{ –insert-clear-stat-by-icount=n}, where n is a positive
integer, which leaves a line with just a ”C” after n lines of instructions.    
\end{itemize}

\THsec{ Configuration}{conf}
Configuration is a list of keys with assigned values that sets out the characteristics
for the CPU to be simulated. Some examples can be the number and sizes of
caches, the sizes of buffers, number of cores, frequency and a lot more. The PROCESSOR
STORE PREFETCH key is the one used to set the store prefetch policy. Another
key to keeping track of is SIMULATION BENCHMARK which holds the name of
the simulation benchmark program so that it can be added to the stats-file (see
\ref{subsec:GEMs}). Between the different simulations, only the values of the two described keys
are changed, the rest remains the same and sets out the architecture of the CPU
which will be described in the following subsection.

\newpage
\THsec{CPU architecture}{arc}
The CPU design simulated for this master thesis (the numbers introduced in figure
\ref{img:arc}) are listed in the table below and are used in similar researches \cite{thePaper}. 

\begin{table}[H]

\label{tbl:config}
%\vspace{-.2cm}
%\renewcommand{\arraystretch}{0.95}
\tabcolsep0.07cm
\centering
\footnotesize
\begin{tabular}{ll}
\hline
\multicolumn{2}{c}{\textbf{Processor (HSW-class)}}\\
\hline
%\hline
%Processor frequency    & 2.0GHz\\
Issue and commit width & 4\\
Instruction queue (IQ)      & 60 entries\\
Reorder buffer (ROB) size               &  192 entries\\
Load queue (LQ)            & 72 entries\\
Store queue (SQ), Store buffer (SB)           & 42 entries\\
Lockdown table (LDT) & 32 entries\\
\hline
\multicolumn{2}{c}{\textbf{Memory}}\\
\hline
Private L1 cache       & 32KB, 8-way, 4 hit cycles\\
Private L2 cache       & 128KB, 8-way, 12 hit cycles\\
Shared L3 cache        & 1MB per bank, 8-way, 35 hit cycles\\
Memory access time     & 160 cycles\\
\hline
\multicolumn{2}{c}{\textbf{Network}}\\
\hline
Topology / routing      & 2D mesh / Deterministic X-Y\\
Data / Control msg size    & 5 / 1 flits\\
Switch-to-switch time   & 6 cycles\\
\hline
\end{tabular}
%\vspace{-.3cm}
\caption{Simulated system configuration}
\end{table}

\THsec{Trace: Interface Sniper-GEMS}{trace}
A CPU cannot read program languages like C or Java directly. Instead, the CPU
has a set of instructions, called instruction set, that it can compute. The instruction
set can differ from one type of CPU to another, but three types of instructions that
can be found in some way in any instruction set are; arithmetical operations such
as, addition, subtraction, multiplication and division, memory instructions such as,
loads and stores, and branch instruction. Branches tell the CPU if it should jump to another
instruction or not. Consider a for-loop, that loop will be represented by a branch
instruction telling the CPU if it should jump back to the beginning of the loop or
continue below it. When compiling a C or Java file, you end up with a binary file
that contains the translation from the source code to a particular instruction set. This file
is the one you use to run the application in question. A trace file is a human-readable
”binary file.”
\\\\
PC stands for program counter, and it keeps track of which instruction in a program
to be executed next. When taking a branch, the CPU changes the PC to the address of the target instructions from a branch and continue forward.
\\\\
The trace files used here will cover more details regarding memory instructions,
that is loads and stores. That is why a trace file has a line break after each memory
instruction. Instructions that are not store instructions nor branches will be referred to as anonymous instructions. Below follows some examples on lines from traces which will be
explained and translated into English.
\begin{figure}[h]
\begin{lstlisting}[frame=single]  
4030fc
\end{lstlisting}
\centering
 \emph{The first line of a trace is the starting PC}
  \caption{Trace example: Starting PC}
\end{figure}

The trace starts with the value of the starting PC in hexadecimal-format. From now on the differentiation of the PC from one instruction to the next is written in decimal-format along with the instruction. Note that the second instruction shows the length
(the change to the PC) of the first one, and so on.

\begin{figure}[h]
\begin{lstlisting}[frame=single]  
0 4 L4 ee7e220 4
\end{lstlisting}
\centering
\emph{Two anonymous instructions, both with length 4 followed by a load of 4
bytes from memory address ee7e220}
  \caption{Trace example: Anonymous instructions and a Load}
\end{figure}

The 0 and the 4 denotes two anonymous instructions (loads, stores and branches
are of greater interest). Since the first digit denotes the size of the previous instruction,
it shows that the two anonymous instructions both have the size of four (the ”4” after
the first space and the ”4” after the ”L”). The ”L” tells us that the instruction is a
load instruction and loads data from the hexadecimal memory address ee7e220. The
number of bytes to be loaded is written after the last space, in this case, four.

\begin{figure}[h]
\begin{lstlisting}[frame=single]  
0d1d3 b4d1t99* S99 7fff8368acd8 8
\end{lstlisting}
\centering
\emph{An anonymous instruction of size 4, that is dependent on the first and the third instruction before this. This instruction is followed by a taken branch to PC+99 ahead with a dependency on the prior instruction. Last is a store of 8 bytes to the address
7fff8368acd8.}
  \caption{Trace example: Branch and store}
\end{figure}

Given the previous examples, this line starts with one anonymous instruction of
size four. In this case, the first instructions are also dependent on the results from the
two previous instructions, and this is denoted by the two ”d” before the first space. The
digit after each ”d” points out how many instructions there are before the dependent one are, in this case one and three. The next instruction starts with a ”b” which tells us that it is a
branch. The branch is dependent on the instruction before, the anonymous one that
was discussed at the beginning of this example. The ”t” denotes the branch target
which afterward is defined as the difference between the current PC and the target address. Here 99 should be added to the PC if taken. The ”*” denotes that the branch is taken. After
that, there is an ”S,” namely a store instruction. Here one can once again see that the
branch is taken since the difference from the branch instruction is 99 (directly after
the ”S”) as the branch target. A store interaction follows the same pattern as a load,
here it stores 8 bytes to the memory address 7fff8368acd8.

\begin{figure}[h]
\begin{lstlisting}[frame=single]  
0m3 b4t-166* L-166 7ff8368acd8 8
\end{lstlisting}
\centering
\emph{An anonymous instruction of size 4 that is dependent on data from memory that is retrieved in the third prior instruction. After that, there is a branch with a
negative PC difference (-166), like the target, which leads to a load of 8 bytes to the
address 7fff8368acd8.}
  \caption{Trace example: Branch to a previous instruction}
\end{figure}

 In this example, there is only have two new things to cover. The ”m” denotes a
memory dependency, some data that are to be loaded in the third previous instruction.
These ”d’s” depend on data to be computed by the instruction in question while the
”m’s” are data that need to be loaded into the CPU. The next thing to cover is that
there is a negative PC difference for the target address. This is all required knowledge
to directly translate the line into English: An anonymous instruction of size 4 that is dependent on
data from memory, retrieved in the third prior instruction. After that, there
is a branch with a negative PC difference (-166) as the target which leads to a load
of 8 bytes from the address 7fff8368acd8.

\THsec{Metrics for evaluation}{meEval}
During this thesis, a python script was written to automatizes the run of the simulation
on every subfolder in a predefined folder (all subfolders have to contain traces
from a benchmark program). In the source code, there is a predefined array with the
 names of the store prefetch policies you want to run. The script will then run GEMS
on every trace with every store prefetch policy and name the stat-file in the following
way: [name on benchmark] [name on store prefetch policy].stats
 \\ \\
The script will also generate the results by reading some values from every stats file,
the values to be read are hard-coded into an array. After collecting all these
values from every file, the values from all benchmarks for every store policy will be
summed. The values for OnCommit will be set to 100\%, and the others will get their
percentages based on that. These percentages will then be plotted into bar charts and written into tables (which
can be found in the results).
\\\\
The average results that you will see in the result chapter \ref{chap:results} is calculated in the following way:

$$\frac{\sum \frac{policy}{OnCommit}}{NumberOfBenchmarks}$$ 

\THsub{Energy graphs}{energy}
Table \ref{tab:energy} lists the energy consumption of accessing our three caches and the network
flits which are the energy needed for transmission of one data unit between caches and
the pipeline. One flits for transmission of control messages and five for data \cite{thePaper}
To retrieve the energy consumption of the caches we use the CACTI-P tool \cite{energy2}. The energy consumption of the caches is used to
estimate the energy consumption of the different caches structures, assuming an 22 nm
technology node. To estimate the dynamic energy consumption of the interconnection
network, we assume that it is proportional to the data transferred \cite{energy} and that each
flit transmitted through the network consumes the same amount of energy as reading
one word from an L1 cache each time that it crosses a link (link and router energy).
These number are used by the python script \ref{sec:meEval} to generate the energy graphs that can be seen in the results chapter \ref{chap:results}.
\begin{table}[h]
    \begin{centering}
        \begin{tabular}{ |c|c| }
            \hline
            CPU part & Energy [nJ] \\ \hline
            Accessing L1 & 0.013343 \\ \hline
            Accessing L2 & 0.0214929 \\ \hline
            Accessing L3 & 0.0454353 \\ \hline
            Network flit & 0.0033357 \\ \hline
        \end{tabular}
        \caption{Energy measurements for different CPU parts.}
        \label{tab:energy}
    \end{centering}
\end{table}







