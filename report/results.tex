\chapter{Results}
\label{chap:results}

This chapter shows the results of the different filters and store prefetch policies concerning; execution times, number of L1 accesses, number of store prefetches and
energy consumption, which are referred to as metrics. This chapter is divided in four
sections, one for each set of techniques (introduced in chapter \ref{chap:ProposedPrefetchPolicies}) and the State-of-the-art ones (see \ref{subsec:GPP}). In each of these sections, there is one subsection for each
metric. Every subsection presents the average values for each compared policy combination in a table. This table follows by a graph with the values of all benchmark
for all compared policy combination. The values are given in percentages normalized
to OnCommit. Finally the five benchmarks that benefit the most and the least from
a certain policy combination compared with OnCommit are presented.
\THsec{State-of-the-art policies}{base}

First, the state-of-the-are store prefetch policies will be analyzed:
\begin{itemize}
    \item NoPrefetch
    \item OnExecute
    \item OnCommit
\end{itemize}
\resExtime
\avgTable{extime}{run1-0-avg}{execution time}{}
This table shows the impact of the prefetched policies on average runtimes. When
using any type of prefetch policy it will roughly cut the runtime with a third (for
OnCommit and OnExecute). These results tell us that bringing data into the L1
cache for store instructions is a bottleneck. Still, it is not just to prefetch data for
all store that may be executed. If comparing OnExecute and OnCommit, one can
see that OnExecute is 2.35 \% units slower. Since OnExecute issues there prefetches earlier they will, therefore, prefetch more data that later will turn out to not be used. Unnecessary prefetch slows down the execution, and should be minimized.
\fullTable{extime}{run1-0-full}{Execution time}
\toplist{extime}{run1-0}{execution time}{OnExexcute}
The table above presents the number of percent units in terms of the number of
cycles for a particular benchmark when comparing OnExecute and OnCommit. Gcc
3 and Gobmk 4 gain over 20 \% in reduced number of cycles while running OnCommit
(being less aggressive with the prefetch). On the other hand, OnCommit is increasing the number of cycles for Omnetpp and CactusADM with a 20 \% units. The ones that
goes well with OnExecute are the ones that do not have so much need of prefetching,
makes the best results for NoPrefetch. The ones that are very sensitive to prefetch
go well with OnExecute.
\resAcc

\avgTable{L1}{run1-1-avg}{number of L1 accesses}{}
This is an interesting table which shows the amount of unnecessary work caused
by prefetching. NoPrefetch does, as the name suggests, no prefetches which mean that
all L1 accesses are caused by store instructions that write to the cache. Everything
above 84.82\% is unnecessary work and will waste energy. OnExecute does as expected
more accesses than OnCommit since it prefetches earlier and is more speculative.

\fullTable{L1}{run1-1-full}{Number of L1 accesses}
\toplist{L1_1}{run1-1}{number of L1 accesses}{OnExexcute}
The table shows that OnCommit decreases the L1 accesses for all but two of
the benchmarks and the difference can be over 90\% units (Bzip2 1 and Bzip2 2).
This behavior shows once again that an early prefetch triggers unneeded accesses to
 the L1 cache. It is interesting that Soplex 1 has more accesses for OnCommit than
OnExecute and this might be something to investigate further.
\resSp
\avgTable{sp}{run1-2-avg}{number of store prefetches}{}
The number of store prefetches confirms the behaviors and conclusions drawn from the tables over execution times and L1 accesses. NoPrefetch does not store prefetches
while OnExeute does it the most. 299.72\% units more than OnCommit. This is also
what to expect since OnExecute is more speculative then OnCommit. Another thing
to notice here is that sometimes after commit, the store instruction is already at the
head of the Store Buffer, and therefore it issues the write instead of the prefetch. The
effect is equivalent with NoPrefetch.
\fullTable{sp}{run1-2-full}{Number of store prefetches}
\toplist{sp1}{run1-2}{number of store prefetches}{OnExexcute}
OnExecute issues more store prefetches then OnCommit for all the benchmarks
except Bwaves which has the same number of prefetches for OnCommit and OnExecute.
\resEnergy
\avgTable{energy}{run1-3-avg}{energy consumption}{}
This table shows the energy consumption of each prefetch technique. First of all,
NoPrefetch consumes the smallest amount of energy while OnExecute consumes the
most. This consumption comes from unnecessary prefetches. It is also worth noting
that if using OnExecute instead of OnCommit the number of cycles decreases by 2.35\%
units, but the energy consumption increases with 33.13\% units. Using OnCommit
instead of NoPrefetch burns 9.05\% units more energy but that gives a speedup of
46.5\% units. A prefetch policy can pay off in decreased energy consumption, but if getting too far in prefetching the loss concerning increased energy consumption can be huge.
\fullTable{energy}{run1-3-full}{Energy consumption}
\toplist{energy}{run1-3}{energy consumption}{OnExexcute}
Using OnCommit rather then OnExexute saves much energy for many benchmarks. Finding Bizp2 in the lead when it comes saving energy is no surprise since it is in the lead when it comes to decreasing prefetches and L1 accesses. This correlation will also explain why Milc increase a bit in energy consumption while using OnCommit instead of OnExecute. Longer execution time will also increase the energy consumption.



\THsec{Techniques to reduce speculation effect}{reduce}
Second, all prefetch policies from state of the art and the following three new ones
will be analyzed:
\begin{itemize}
    \item OnNonBSpec (OnNonBSpeculative)
    \item OnExecute with Re-Execute
    \item OnNonBSpec with Re-Execute
\end{itemize}
\resExtime
\avgTable{extime2}{run2-0-avg}{execution time}{}
OnNonBSpec seems to be the fastest one, 3.34\% units faster then OnCommit.
It makes a difference not to prefetch stores that are affected by a branch. The Re-Execute filter affects OnExecute with 1.22\% units but OnNonBSpec with only 0.08\%
units.
\fullTable{extime2}{run2-0-full}{Execution time}
\toplist{extime2}{run2-0}{execution time}{OnNonBSpec with Re-Execute}
Most of the benchmarks benefit from using OnNonBSpec with Re-Execute, in the
lead, CactusADM (with 14.97\% units) and Omnetpp (with 11.05\% units). The ones
that loses by using OnNonBSpec with Re-Execute does so by under two percent.
\resAcc
\avgTable{L12}{run2-1-avg}{number of L1 accesses}{}
 OnNonBSpec does 4.04\% units less L1 accesses then OnExectue. Re-Execute decreases the L1 accesses with 3.60\% units used on OnExecute and 0.95\% units used on
 OnNonBSpec. The difference can be explained by OnNonBSpec removing some of the
prefetches that ReExecute also will remove. The total numbers tell us that OnNonBSpec with Re-Execute has the lowest number of L1 accesses apart from OnCommit
 with 100.00\% against 118.26\% (OnExectue with Re-Execute). 
\fullTable{L12}{run2-1-full}{Number of L1 accesses}
\toplist{L12}{run2-1}{number of L1 accesses }{OnNonBSpec with Re-Execute}
In this table OnBSpec has more L1 accesses for all benchmarks except Spolex.
Bzip2 1 and Bzip2 2 have over ninety percent more accesses and are the two benchmarks that have the highest increase.
\resSp
\avgTable{sp2}{run2-2-avg}{number of store prefetches}{}
OnNonBSpec decreases the number of store prefetches by 57.60\% units compared
with OnExecute. Re-Execute is also decreasing the number of prefetches with 15.64\%
units used on OnExecute and 7.68\% units used on OnNonBSpec. The difference can
be explained by OnNonBSpec removing some of the prefetches that Re-Execute also
will remove. The total numbers tell us that OnNonBSpec with Re-Execute has the
lowest number of L1 accesses apart from OnCommit (and NoPrefetch) with 336.54\%
against 384.08\% (OnExectue with Re-Execute).
\fullTable{sp2}{run2-2-full}{Number of store prefetches}
\toplist{sp2}{run2-2}{number of store prefetches}{OnNonBSpec with Re-Execute}
OnNonBSpec with Re-Execute does more prefetches then OnCommit for all benchmarks except Bwaves, which stays the same. In first place is bzip2 1 which increases the prefetch by 4687 percent units.
\resEnergy 
\avgTable{energy2}{run2-3-avg}{Energy consumption}{}
What to notice here is that OnNonBSpec burns 22.89\% units less energy then
OnExecute. Re-Execute decrease the energy consumption with 3.24\% units used on
OnExecute and 0.82\% units used on OnNonBSpec. OnNonSPec with ReExecute gets
109.42\% units, which is 9.42\% units more than OnCommit.
\fullTable{energy2}{run2-3-full}{Energy consumption}
\toplist{energy2}{run2-3}{energy consumption}{OnNonBSpec with ReExecute}
OnNonBSpec with Re-Execute can decrease the energy consumption for some
benchmarks with only a few percent units, and the best is Gcc 2 with 4.37 \% units
less energy consumption. However, for some other benchmarks like Bzip2 1 and
Bzip2 2, the consumption is increased, in this case by around 90 \% units. To see
these two benchmarks consume the most energy is not a surprise since they are high
on L1 accesses and store prefetch.

\subsection{Conclusion}
OnNonBSpec with Re-Execute is the fastest store prefetch policy with a speed-up of 3.42 \% units compared to OnCommit, but it burns 9.42 \% units more energy. The
question to ask is if an energy increase with 2.75 \% units is worth a speed-up of 1 \%
units. It is likely to believe that the answer will differ with the type of machine and
application. OnNonBSpec with Re-Execute will be kept together with the three stateof-the-art policies as the reference for the future once. The new filters and policies to
be tested will all be put on top of it. All the polices (except NoPrefech) still consumes
more energy then OnCommit, the techniques introduced next will aim to reduce this
consumption.
\THsec{Techniques to filter unnecessary prefetches}{filter}
In this section sameCachLine and PCbasedPredictor (PCbased) are introduced. The  digit after PCbased denotes the buffer size, the number of entries is two to the power
of the digit. Tests have been conducted with PCBased2, 4, 6, 8, 10 and 16 but as it
turns out 2 behaves in the same way as 4, and 8 the same as 6, and so on in all four
metrics (execution time, L1 accesses, number of prefetch and energy consumption).
Therefore, to reduce the number of configurations in the tables of this section, only
buffer sizes 2 and 8 are used. The new configurations, based upon OnNonBSPec with
Re-Execute, are:
\begin{itemize}
    \item OnNonBSpec with sameCacheLine
    \item PCbasedPredictor 4
    \item PCbasedPredictor 4 with sameCacheLine
    \item PCbasedPredictor 8
    \item PCbasedPredictor 8 with sameCacheLine
\end{itemize}
\resExtime
\avgTable{extime3}{run3-0-avg}{execution time}{* with sameCacheLine}
 SameCacheLine has a little slowdown on OnNonBSpec (the best from 5.2) with
0,06\% units and 0.08\% units, on PCbasedPredictor (2 and 8). One hypothesis for this
is that loading the same permissions twice in a short manner of time will not be an
issue since the memory-subsystem will quickly realize that the requested permission
is already in the L1 cache. Therefore blocking the second prefetch of a cache line a
bit earlier will not have a big impact, at least on the execution time.
\\ \\ 
PCbased 4 and 8 have the same numbers 96.96\% without sameCacheLine and 96.88\% with. These results can be compared to OnNonBSpec with its 96.64\%, with sameCacheLine. Iterate over a buffer as in PCBasedtakes time, so the advantage of doing
that has to pay more at runtime. Otherwise, it will end up with a slowdown as here.

\fullTable{extime3}{run3-0-full}{Execution time}
\toplist{extime3}{run3-0}{execution times}{OnNonBSpec with SameChacheLine}
Here OnCommit is compared with OnNonBSpec with Re-Execute and SameCacheLine, since its the best one concerning execution time among the new one for
this section. Some benchmarks seem to benefit from OnNonBSpec with Re-Execute and SameCacheLine while others do not. The absolute value of the one benefiting
the most from OnNonBSpec with Re-Execute and SameCacheLine is higher than the
one losing the most ($| − 16.10| > |6.58|$).
\resAcc
\avgTable{L13}{run3-1-avg}{number of L1 accesses}{* with sameCacheLine}
PCbasedPredictor 4 and 8 have the same number of L1 accesses (105.76\%). SameCacheLine improves both on OnNonBSpec with 1.15\% units and on PCbasedPredictor (4 and 8) with 0.73\% units. PCbasedPredictor (4 and 8) with SameCacheLine
becomes the closest one to OnCommit with it is 105.03\%.
\fullTable{L13}{run3-1-full}{Number of L1 accesses}
\toplist{L13}{run3-1}{Reduce speculation the benchmarks which number of L1 accesses are affected the most (in number of percent)}{OnNonBSpec with SameCaceLine}
This table shows us that OnNonBSpec with Re-Execute and SameCacheLine decreases the L1 accesses for only two benchmarks, Milc (-3.52\% units) and Gcc 2 (0.17\% units). Hmmer is the benchmark that has the greatest decrease of L1 accesses
26.20\% units.
\resSp
\avgTable{sp3}{run3-2-avg}{number of store prefetches}{* with SameCacheLine}
The store prefetches shows roughly the same thing as L1 accesses (see table 5.19).
PCNessary (4 and 8) gives the same percentage without SameChaheLine, 155.16\%,
and with, 151.95\%, a difference of 3.21\% units. SameCacheLine also improves OnNonBespec with 6.02\% units. PCNessary (4 and 8) with SameCacheLine is once again
closest to OnCommit with it is 151.95\%.
\fullTable{sp3}{run3-2-full}{Number of store prefetches}
\toplist{sp3}{run3-2}{number of store prefetches}{OnNonBSpec with SameChacheLine}
All benchmarks do more store prefetches then OnCommit when using OnNonBSpec with Re-Execute and SameCacheLine, except for Bwaves which stays the same.
The two benchmarks that increased their store prefetches the most are Libquantum
(247.48 percent units) and Bzip2 5 (240.84 percent units).
\resEnergy
\avgTable{energy3}{run3-3-avg}{energy consumption}{* with sameCacheLine}
There is no surprise that the energy corporation looks the same as L1 access and
store prefetches. PCNessary (4 and 8) has the same percentages, without, 103.91\%,
and with SameCacheLine and it is decreased with 0.66\% units to 103.25\%. OnNonBSpec drops with 0.78\% units to 108.64\% with SamecacheLine. However, it is not
enough to beat PCBasedwith SameCacheLine.
\fullTable{energy3}{run3-3-full}{Energy consumption}
\toplist{energy3}{run3-3}{energy consumption}{OnNonBSpec with SameCacheLine}
 Four of our benchmarks saves energy by using OnNonBSpec with Re-Execute and
SameCacheLine. The one saving the most is Milc with its 4.47\% units. Hmmer does 
must store prefetches and consumes the must energy, 22.87\% units more than for
OnCommit.
\subsection{Conclusion}
SameCacheLine does not get affect at all on the execution time, but it saves energy
by removing L1 accesses and prefetches. Therefore SameCachLine should be added
to the currently best policies. PCbasedPredictor gives the same runtime and burns
the same amount of energy for both sizes 4 and 8. It is cheaper and more efficient
to have a buffer with $2^4=16$ entries then $2^8=256$ entries and therefore PCBased4
does and OnNonBSpec, both with SameCacheLine will be compared.

    \begin{table}[H]
\centering

\begin{tabular}{ |l|l|l| }
\cline{2-3}
\multicolumn{1}{ c| }{} 
& Execution time & Energy  \\  \hline
PCBased 4& 96.88 & 103.25  \\  \hline
OnNonBSPec & 96.64 & 108.64  \\  \hline
Difference & 0.24 & 5.39  \\  \hline
\end{tabular}
\caption{Comparation between PCBased 4 and OnNonBSpec, both with SamecacheLine. }
\label{quickcompPC4vsONBS}
\end{table}
In the table above \ref{quickcompPC4vsONBS} shows that OnNonBSpec is fastest but PCNessary 4 burns
less energy. So the choice is based on what the biggest concern might be in a given setting, time or energy. PCbasedPredictor 4 is a bit slower, 0.24\% units, but the energy
savings is much higher (5.29\% units) therefore should PCNessary 4 with SameCacheLine be considered the best policy of this section and be together with the basic three
the ones to be taken to the next section.
\THsec{Techniques for timelLiness}{timelines}
In this last section the following policies are added, which all build upon PCbasedPredictor 4 with Re-Execute and SameCacheLine:
\begin{itemize}
    \item PCbasedPredictorTimeliness2 4 (PCbasedTL2 4)
    \item PCbasedPredictorTimeliness2 8 (PCbasedTL2 8)
    \item PCbasedPredictorTimeliness3 4 (PCbasedTL3 4)
    \item PCbasedPredictorTimeliness3 8 (PCbasedTL3 8)
\end{itemize}
\resExtime

\avgTable{extime4}{run4-0-avg}{execution time}{* with reExecute and SameCacheLine}
PCNessaryTimeliness3 has longer run times, 97.89\% for size 4 and 97.51\% for size 8,
iterate over a bigger buffer takes more time for PCNessaryTimeliness3.
\fullTable{extime4}{run4-0-full}{Execution time}
\toplist{extime4}{run4-0}{execution time}{PCbasedPredictor~4*}
PCbasedPredictor 4 speeds up at least five benchmarks with Mlic in the top (7,84\%
units). On the other hand, some benchmarks get higher execution times, Gcc 5 is
increasing the most with 2.40\% units.
\resAcc
\avgTable{L14}{run4-1-avg}{number of L1 accesses}{* with reExecute and SameCacheLine}
PCNessaryTimeliness2 (4 and 8) gets the same number of L1 accesses as PCNessary 4, 105.03\%. PCNessaryTimeliness3 (4 and 8) are getting more or less the same precentages, (98.36\% and 98.34\%), pretty close to OnCommit, 100.0\%
\fullTable{L14}{run4-1-full}{Number of L1 accesses}
\toplist{L14}{run4-1}{number of L1 accesses}{PCbasedPredictor~4*}
 These results are pretty interesting. Both tables have roughly the same absolute
values. One benchmark can decrease the L1 accesses with the same amount that
another benchmark increases its L1 accesses. Milc decreases with 4.30\% units by
using PCbasedPredictor 4 instead of OnCommit, while Lbm increases by 4.53\% units.
\resSp
\avgTable{sp4}{run4-2-avg}{number of store prefetches}{* with reExecute and SameCacheLine}
Once again PCNessaryTimeliness2 (4 and 8) gets the same number of L1 accesses as PCNessary 4, 151.95\% and PCNessaryTimeliness3 lays close to OnCommit, (99.60\% for size 4 and 98.89\% for size 8).
\fullTable{sp4}{run4-2-full}{Number of store prefetches}
\toplist{sp4}{run4-2}{number of store prefetches}{PCbasedPredictor~4*}
PCbasedPrediction is reducing the number of store prefetches for hmmer 1 with
10.01\% units but increases the amount for Bzip2 5 with 22.24\% units. The absolute values for the increased ones are higher than for the reduced ones (i.e., $|22.24| > | − 10.01|$).
\resEnergy
\avgTable{energy4}{run4-3-avg}{energy consumption}{* with reExecute and SameCacheLine}
Once again PCNessaryTimeliness2 (4 and 8) gets the same energy consumption as
PCNessary 4, 103.25\%. However, PCNessaryTimeliness3 8 consumes 5.09\% units less
energy than OnCommit and is, therefore, the prefetch policy that burns less energy than OnCommit,  100.08\%.
  
\fullTable{energy4}{run4-3-full}{Energy consumption}
\toplist{energy4}{run4-3}{energy consumption}{PCbasedPredictor~4*}
PCbasedPredictor 4 have higher absolute values when it comes to the benchmarks with decreased energy consumption. Milc saves most energy of the benchmarks when comparing between OnCommit and PCbasedPredictor 4 with its 5.04\% units.
Gamess 2 consumes 3.01\% units more energy using PCbasedPredictor 4 instead of
OnCommit.

\subsection{Conclusion}
PCNessaryTimeliness2 (4 and 8) are out of the game since they perform the same as PCbasedPredictor 4 in all measured aspects. PCNessaryTimeliness3 8 is interesting since it consumes less energy then OnCommit and is also faster. It is 2.49\% units
faster then PCbasedPredictor 4, but it burns 1.84\% units more energy. Therefore
PCNessaryTimeliness3 8 is the best prefetch policy proposed in this thesis.

\section{Combined results}
Below you will once again find all four aspects (Execution time, L1 accsess, Store Prefetch, Energy Consumption) for all policies.
    \begin{table}[H]
\centering

\begin{tabular}{ |l||l|l|l|l| }
\hline
Policy&Execution time & L1 accsess & Store Prefetch & Energy Consumption  \\  \hline \hline
NoPrefetch & 146.50 & 84.25 & 0.00 & 90.95 \\ \hline
OnExecute & 102.35 & 121.86 & 399.72 & 133.13 \\ \hline
OnCommit & 100.00 & 100.00 & 100.00 & 100.00 \\ \hline
OnNonBSpec & 96.66 & 113.82 & 342.12 & 110.24 \\ \hline
OnExecute* & 100.57 & 118.26 & 384.08 & 129.81 \\ \hline
OnNonBSpec* & 96.58 & 112.87 & 336.54 & 109.4 \\ \hline
PCbased 4* & 96.96 & 105.76 & 155.16 & 103.91 \\ \hline
PCbased 4** & 96.88 & 105.03 & 151.95 & 103.25 \\ \hline
PCbased 8* & 96.96 & 105.76 & 155.16 & 103.91 \\ \hline
PCbased 8** & 96.88 & 105.03 & 151.95 & 103.25 \\ \hline
PCbasedTL2 4** & 96.88 & 105.03 & 151.95 & 103.25 \\ \hline
PCbasedTL2 8** & 96.88 & 105.03 & 151.95 & 103.25 \\ \hline
PCbasedTL3 4** & 97.89 & 98.36 & 99.60 & 98.24 \\ \hline
PCbasedTL3 8** & 97.51 & 98.34 & 98.89 & 98.16\\ \hline
 \end{tabular}
 * with Re-Execute \\
**with Re-Execute and SameCacheLine
\caption{The average numbers for all policies for all four aspects}
\label{Com}
\end{table}