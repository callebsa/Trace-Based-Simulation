\chapter*{Sammanfattning}
Detta arbete kommer att fokusera på hur prestandan vid körning av program på en
CPU kan ökas. Mer specifikt kommer store instruktionerna att studeras. Minnes
instruktioner osakar ofta stora förseningar i samband med väntan på att data ska
överföras från RAM-minnet till L1 cachen. Om en out-of-order CPU inte kan hitta
andre instruktioner att jobba med i väntan på datan så kommer dessa cyklar att
slösas bort. För att försöka överkomma denna problematik så har min handledare,
Alberto Ros Bardisa, kommit med iden att använda predictors för att förutspå vilken
data som kommer att användas inom snar framtid och överföra den till L1 cachen i
förväg. Detta liknar branch predictors som används i flera år, där CPUn matas med
instruktioner att börjar jobba med baserat på en gissning om branchen kommer att tas
eller inte. Här gissar vi vilken data som ska överföras till L1 i förväg istället för vilken
instruktioner som kommer efter en branch. Såklart är det önskvärt att gissningen är
korrekt, men även om den är det så kan vi ha problem: När ska datan överföras? För
sent och vi måste fortfarande vänta eftersom behovet av data uppkommer ni den är
på plats. För tidigt och data kan bli avlägsnad från L1 cachen på grund av platsbrist
och det blir en väntetid för att överföra datan på nytt när behovet väl uppstår. De
senare är också ett slöseri av energi då vi överför något som kommer att avlägsnas
innan användning. Frågan i detta masterarbete är när data ska överföras för optimal
prestanda.
\newpage