\chapter*{Sammanfattning}
Detta arbete kommer att fokusera på hur prestandan vid körning av program på
en CPU kan ökas. Mer specifikt kommer Store instruktionerna att studeras. Dess
instruktioner osakar ofta stora förseningar i samband med väntan på att data ska
överföras från RAM-minnet till L1 Cachen. Om en out-of-order CPU inte kan hitta
andra instruktioner att jobba med i väntan på datan så kommer dessa cyklar att
bortkastas. För att försöka överkomma denna problematik är en ide att använda
Predictors för att förutspå vilken data som kommer att användas inom en snar framtid
och överföra den till L1 Cachen i förväg. Detta liknar Branch Predictors där CPUn
matas med instruktioner att börjar jobba med baserat på en gissning om branchen
kommer att tas eller inte. Här gissar vi vilka skrivrättigheter som ska överföras till L1 i förväg
istället för vilka instruktioner som kommer efter en Branch. Det är självklart önskvärt
att gissningen är korrekt, men även om den är det så kan vi ha problem, och det är
när ska datan överföras. För sent och vi måste fortfarande vänta eftersom behovet
av skrivrättigheten uppkommer ni den är på plats. För tidigt och skrivrättigheten kan bli avlägsnad från L1
cachen på grund av platsbrist samt att det blir en väntetid för att överföra datan på
nytt när behovet väl uppstår. De senare är också ett slöseri av energi då vi överför
något som kommer att avlägsnas innan användning. Målet med detta masterarbete
är att utröna när data ska överföras för optimal prestanda.
\newpage