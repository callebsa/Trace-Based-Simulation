\chapter{Discussion}
\label{chap:discussion}

\paragraph{Energy calculations -} The calculation of energy consumption has been implemented during the work with this thesis \fixme. It is computed based on the network flits as well as the number of accesses the caches. The problem is that by implementing more and more advanced prefetch policies more logic gates are added and this will of course also burn energy. I have discussed the issue of added new buffers hat will be read and witted many times during the execution, as for PCNecessary and PCNecessaryTimliness, with my supervisor. We believe that this unmeasured consume as little in comparison with the values that are measured. This is something to keep in mined while comparing more or less complex policies. 

\paragraph{Simulation failures -} All filters and policies apart form the three state-of-art once and OnNonBSpec was implemented within this thesis. The way it was done was based by an idea about how more unwanted prefetches can be excluded from the wanted once. Then having an idea about the theoretical impact it was implemented and tested. The result was then compared with our expectations and if the somewhat matches, say the new policy is getting a lower execution time the an old one, we were happy with it. If not we was rethinking on the theoretical impact and conducted tests on very short handmade traces so that the simulation can be monitored manually step by step in order to understand why the result do not match our expectation. I can see two risks with this approach. One, if we have theory was wrong, then a correct (or more correct) implementation might be changed for a more faulty once in order to much the wrong theory. Two, given a correct theory they might still be bugs cousing the values to be wrong but it will not be noticed since it points in the same direction as the theory. The new policy have shorter execution time then an old on but it might be even shorter or not that mush shorter.