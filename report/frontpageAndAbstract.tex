% MUST use a4paper option
% MAY use twoside, smaller font, and other class
\documentclass[a4paper,12pt]{article}
% Use UTF-8 encoding in input files
\usepackage[utf8]{inputenc}
% If you are writing in English, un-comment the following line:
\usepackage[swedish,english]{babel}
% Use the template for thesis reports
\usepackage{UppsalaExjobb}

%for formating
\usepackage{etoolbox}


% Designval: per default används styckesindrag, men ibland blir det snyggare/mer lättläst med tomrad mellan stycken. Det åstadkoms av de följande raderna.
% Tycker ni om styckesindrag mera, kommentera bort nästa två rader.
\parskip=0.8em
\parindent=0mm

%\preto\paragraph{\filbreak}

\begin{document}
% Set title, and subtitle if you have one
\title{Automatic reading and interpretation of paper invoices}
\subtitle{ADC invoice} 
% Set author names, separated by "\\ " (don't forget the space)
\author{Boström, Carl}
% Set the date and year - use the right language!
\date{May 2016}
% Only need to set the year if it differs from the current year
%\year=2016

% Ange handledare, ämnesgranskare, examinator om dessa finns
% Handledare: t.ex på företag ni arbetat med?
\handledare{Crowderia}
% Ämnesgranskare används inte på Självständigt arbete i IT
%\reviewer{NN}
% På Självständigt arbete i IT är detta BV

\handledare{Sofia Cassel och Björn Victor}
\examinator{Björn Victor}
% OM NI HAFT EXTERN HANDLEDARE
\exthandledare{Håkan Lundmark, Crowderia AB}

\progname{Civilingenj{\"o}rsprogrammet i informationsteknologi}{Computer and Information Engineering Programme}
\enhetsnamn{Institutionen för \\ informationsteknologi}
\besoksadress{ITC, Polacksbacken\\ Lägerhyddsvägen 2}
\postadress{Box 337 \\ 751 05 Uppsala}
\hemsida{http:/www.it.uu.se}

% Programnamn på svenska och engelska
\progname{Computer and Information Engineering Programme}{Civilingenj{\"o}rsprogrammet i informationsteknologi}
% Set the name of the series, and the number in the series
\seriesname{Independent Project in Information Engineering}

% Get a series number from Studentservice Ångström
%\seriesnumber{UPTEC IT16~0xx}
% Use the appropriate ISSN for the series
%\issn{ISSN 1401-5749}
% Usually this is where it is printed
%\printer{Ångströmlaboratoriet, Uppsala universitet}

% This creates the title page
\maketitle

% Change to frontmatter style (e.g. roman page numbers)
\frontmatter

\begin{abstract}

Manually typing long numbers on paper invoices is tedious and time consuming work. The task of typing all the fields of an invoice into a text editor was given to two subjects working with bookkeeping, and the average time consumed was measured to be five minutes. The time and cost spent on manual typing will accumulate for companies that receive a lot of invoices. Swedbank~\cite{Swedbank_app} along with other banks, have addressed this issue with a mobile application that reads and interprets the numbers on an invoice using the built in camera. This solution is directed to the public and the extracted information cannot be imported into bookkeeping software.   A standalone software for digital reading and interpretation of scanned invoices is our solution  for companies in regards to this issue. \\

%Manually typing long numbers on paper invoices is tedious and time consuming work. The task of typing in all field of an invoice to a text editor was giving too same people working with bookkeeping. The time was measured to be on average 5 minutes. The time spent and cost for it will be accumulated for companies receiving a bunch of invoice. Swedbank~\cite{Swedbank_app} along with other banks, has addressed this issue with a mobile application that can read and interpreted the numbers on a invoice using the built in camera. This solution is directed to private people and the extracted information will be send to the bank and cannot be  imported into a bookkeeping software. A standalone software for digital reading and interpenetration of scanned invoices is our solution to companies for this issue.% 

%Crowderia describes a problem with the process of digitizing their invoices. The information on the invoices has to be entered into their digital database by hand. The company gets a substantial amount of invoices every month and these take a large amount of working hours and resources to manually enter into the digital system. Thus they are in search of a software solution that can be used to read the invoices automatically. 
There is already technology available that will interpret written text. These techniques were applied in our work, but the focus of this project was to implement algorithms for finding the location for a specific number and resolve what bookkepping terms the number references e.g. OCR, IBAN numbers, etc. This hampered by the missing of a general invoice layout.
%The issue was addressed by utilizing libraries to scan the analogue writings on the invoices. There is already fairly strong technology available that will interpret written text. These techniques were adapted to fit the setup that is displayed on the invoices, meaning the specific locations of different parts of written information on the paper. The program splits up this information into smaller chunks, and proceeds to search for the parts that are most relevant to this purpose; the OCR, the IBAN numbers, etc. 
\\ 
52\% of all the sought information was extracted correctly and almost 95\% of the bookkeeping details that are changed most frequently on invoices from the same service providers were extracted correctly. The average time it takes for the application to extract the vital data is 30-40 seconds.%, which is considerably less time required than extracting it manually.
%The time and accuracy for ADC Invoice will differ depending on invoice layout. 
\end{abstract}



\end{document}
