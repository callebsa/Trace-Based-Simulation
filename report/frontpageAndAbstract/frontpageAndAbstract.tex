% MUST use a4paper option
% MAY use twoside, smaller font, and other class
\documentclass[a4paper,12pt]{article}
% Use UTF-8 encoding in input files
\usepackage[utf8]{inputenc}
% If you are writing in English, un-comment the following line:
\usepackage[swedish,english]{babel}
% Use the template for thesis reports
\usepackage{UppsalaExjobb}

%for formating
\usepackage{etoolbox}


% Designval: per default används styckesindrag, men ibland blir det snyggare/mer lättläst med tomrad mellan stycken. Det åstadkoms av de följande raderna.
% Tycker ni om styckesindrag mera, kommentera bort nästa två rader.
\parskip=0.8em
\parindent=0mm

%\preto\paragraph{\filbreak}

%\begin{document}
% Set title, and subtitle if you have one
\title{Store prefetch policies}
\subtitle{Analysis and new proposals} 
% Set author names, separated by "\\ " (don't forget the space)
\author{Boström, Carl}
% Set the date and year - use the right language!
\date{September 2018}
% Only need to set the year if it differs from the current year
%\year=2016

% Ange handledare, ämnesgranskare, examinator om dessa finns
% Handledare: t.ex på företag ni arbetat med?

% Ämnesgranskare används inte på Självständigt arbete i IT
\reviewer{Stefanos Kaxiras}
% På Självständigt arbete i IT är detta BV

\handledare{Alberto Ros}
\examinator{Lars-Åke Nordén}
% OM NI HAFT EXTERN HANDLEDARE
%\exthandledare{Håkan Lundmark, Crowderia AB}

\progname{Civilingenj{\"o}rsprogrammet i informationsteknologi}{Computer and Information Engineering Programme}
\enhetsnamn{Institutionen för \\ informationsteknologi}
\besoksadress{ITC, Polacksbacken\\ Lägerhyddsvägen 2}
\postadress{Box 337 \\ 751 05 Uppsala}
\hemsida{http:/www.it.uu.se}

% Programnamn på svenska och engelska
\progname{Computer and Information Engineering Programme}{Civilingenj{\"o}rsprogrammet i informationsteknologi}
% Set the name of the series, and the number in the series
%\seriesname{Independent Project in Information Engineering}

% Get a series number from Studentservice Ångström
%\seriesnumber{UPTEC IT16~0xx}
% Use the appropriate ISSN for the series
%\issn{ISSN 1401-5749}
% Usually this is where it is printed
%\printer{Ångströmlaboratoriet, Uppsala universitet}

% This creates the title page

\begin{document}
\maketitle
% Change to frontmatter style (e.g. roman page numbers)
\frontmatter
\begin{abstract}
This thesis is focusing on how to gain performance when executing programs on a CPU. More specifically the store instructions are studied. These instructions often cause huge delays while waiting on write permission for a particular data block. If an out-of-order (OoO) CPU fills up the entire store buffer with stores waiting for write permission, the CPU has to stall, and cycles are going to be wasted. To over overcome this issue, the idea is to use predictors to try to predict which write permissions that are needed in the future and brought it to the L1 cache in advance. This method is similar to that of the branch predictor where the CPU is fed with instructions to start working in advance based on the prediction of whether or not a branch is to be taken. Naturally, you would want the prediction to be correct, but even if we assume that the predictor always grants needed write permission ahead of time there might still be a problem, and that is when to grant store permission for the data block. Too late and you still need to wait since the need for it occurs before it is in place. Too early and it is going to be evicted from the L1 cache due to space issues, and there are going to be an idle time bring it back in when the need occurs. Furthermore, it is also a waste of energy since we brought in something to be evicted before use. The question I aim to answer with this master thesis is when to prefetch data permission to gain optimal performance.
\end{abstract}



\end{document}
